\documentclass[10pt,a4paper]{article}
\usepackage[latin1]{inputenc}
\usepackage{amsmath}
\usepackage{amsfonts}
\usepackage{amssymb}
\usepackage{graphicx}
\author{Yicheng Liu}
\title{Proof of simplification}
\begin{document}
The flow network $\mathcal{N_\lambda}$ constructed in Section 4 can further be simplified, in an equivalent way. The new network is denoted by $\mathcal{N_\lambda'}$. Some notations are reused in the definition of the new network. All these reused notations corresponds to those with the same notations in $\mathcal{N_\lambda}$. $E,V,V^E,V^M,V^C,C_i,C,\lambda$ hold the same meanings as previous. $\mathcal{N_\lambda'}$ is constructed as follows.

$\mathcal{N_\lambda'}$ consists of 4 types of vertices $s$, $t$, $\{a_{C_i}|C_i\in C\}$ and $\{b_i'|v_i\in V^M\}$. The flow starts from $s$ and ends at $t$. The edges consists of four types as follows.
\begin{enumerate}
	\item For any $C_i\in C$, there is an edge from $s$ to $a_{C_i}$ with capacity $c_\lambda'(s,a_{C_i})=|C_i|*\lambda$.
	\item For any $C_i\in C$, there is an edge from $a_{C_i}$ to $t$, with capacity $|C_i|-1$.
	\item For any $v_kv_j\in E$ satisfying $v_k\in C_i$ and $v_j\in V^M$, there is an edge from $a_{C_i}$ to $b_j'$ with capacity 1.
	\item For any $v_j\in V^M$, there is an edge from $b_j'$ to $t$ with capacity 1.
\end{enumerate}

Same as before, we conduct parametric flow on the new network. Then, the utility profile is decided. If $v_i\in V^E\cup V^M$, $u_i=1$; otherwise, if $v_k\in C_i \subset C$, $u_k=f(s,a_{C_i})/|C_i|$. Here, $f(s,a_{C_i})$ denotes the flow from $s$ to $a_{C_i}$ after the parametric flow algorithm. Actually,  $u_k$ is the value of $\lambda$ where the inflow of $a_{C_i}$ stops increasing. 

\begin{theorem}
	The utility profile yields by the construction above is the same as the utility profile of the original water-filling algorithm.
\end{theorem}

This proof is conducted as follows. We prove that for any feasible final flow assignment\footnote{The final flow assignment means the flow assignment when the algorithm terminates.} of $\mathcal{N_\lambda}$, we can find a flow assignment of $\mathcal{N_\lambda}'$ that yields the same utility profile. Conversely, any feasible final flow assignment of $\mathcal{N_\lambda}'$ also corresponds to a flow assignment of $\mathcal{N_\lambda}$ with the same utility profile.
This theorem is proved by proving the following two statements: (1) the final flow assignment of $\mathcal{N_\lambda}$ corresponds to a state of $\mathcal{N_\lambda}'$ and (2) the final flow assignment of $\mathcal{N_\lambda}'$ corresponds to a state of $\mathcal{N_\lambda}$. (1) and (2) indicate that there is a one-on-one mapping between $\mathcal{N_\lambda}$ and $\mathcal{N_\lambda}'$. A state refers to the network with a flow assignment.

First, (1) is trivial,  

\begin{enumerate}
	\item For any $v_i\in V^E\cup V^M$, remove $a_i$, $b_i$ and the edges connected to them. The utility of $v_i$ is 1.
	\item For each component $C_i\subset V^C$, in our original construction, we have $A^{C_i}$ and $B^{C_i}$ for it. However, we can replace these two parts with a single vertex
\end{enumerate}
First, remove parts $A^{EM}$ and $B^{EM}$ and set final utility of each vertex
in both sets to be 1. Second, for components that are not connected to
M, calculate the utility of each vertex directly as follows: if the size of a
component is $s$, then the utility of any vertex in this component is $\frac{s-1}{s}$. Third, $s$
replace nodes of the same component by a single node, and add up all the capacities and flows between this component and outside nodes accordingly. One can show the simplified network is equivalent to the one constructed in Section 4.

\end{document}